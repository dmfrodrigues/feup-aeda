\documentclass{beamer}
% Encodings (to render letters with diacritics and special characters)
\usepackage[utf8]{inputenc}
% Language
\usepackage[portuguese]{babel}

\usetheme{Madrid}
\usecolortheme{default}

\pdfstringdefDisableCommands{
  \def\\{}
  \def\texttt#1{<#1>}
}

\newcommand{\email}[1]{
{\footnotesize \texttt{\href{mailto:#1}{#1}} }
}

\usepackage[style=british]{csquotes}

\usepackage{tabularx}
 
%Information to be included in the title page:
\AtBeginDocument{
\title[Tema 5 (Parte 1)]{Empresa de Transporte de Mercadorias (Parte 1)}
\subtitle{Transportes SML}
\author[T5G3]{
\begin{tabular}{r l}
	\email{up201806429@fe.up.pt} & Diogo Miguel Ferreira Rodrigues        \\
	\email{up201806554@fe.up.pt} & Telmo Alexandre Espirito Santo Baptista\\
	\email{up201306340@fe.up.pt} & Luís Paulo da Rocha Miranda
\end{tabular}
}
\institute[FEUP/AEDA]{Faculdade de Engenharia da Universidade do Porto \\ Algoritmos e Estruturas de Dados (AEDA) - Turma 5, grupo 3} 
\date[16/nov/2019]{16 de novembro de 2019}
}

\begin{document}
\frame{\titlepage}

\begin{frame}
\frametitle{Problema: Empresa de Transporte de Mercadorias}
\begin{quote}
Modelar um problema recorrendo ao paradigma da orientação por objetos e usar a linguagem C++ para implementar a solução correspondente.
\end{quote}
\begin{quote}
A empresa Transportes SML é especialista no transporte de mercadorias. A empresa possui um número fixo de camiões de diferentes tipos, específicos para o transporte de determinada mercadoria...
\end{quote}
\end{frame}
 
\end{document}

