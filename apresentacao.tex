\documentclass{beamer}
% Encodings (to render letters with diacritics and special characters)
\usepackage[utf8]{inputenc}
% Language
\usepackage[portuguese]{babel}

\usetheme{Madrid}
\usecolortheme{default}

\pdfstringdefDisableCommands{
  \def\\{}
  \def\texttt#1{<#1>}
}

\newcommand{\email}[1]{
{\footnotesize \texttt{\href{mailto:#1}{#1}} }
}

\usepackage{listings}
\lstset{
    frame=tb, % draw frame at top and bottom of the code
    tabsize=4, % tab space width
    numbers=left, % display line numbers on the left
	showstringspaces=false, % don't mark spaces in strings    
%    aboveskip=-2em,
%    belowskip=-1.5
}

\usepackage[style=british]{csquotes}

\usepackage{tabularx}
 
%Information to be included in the title page:
\AtBeginDocument{
\title[Tema 5 (Parte 1)]{Empresa de Transporte de Mercadorias (Parte 1)}
\subtitle{Transportes SML}
\author[T5G3]{
\begin{tabular}{r l}
	\email{up201806429@fe.up.pt} & Diogo Miguel Ferreira Rodrigues        \\
	\email{up201806554@fe.up.pt} & Telmo Alexandre Espirito Santo Baptista\\
	\email{up201306340@fe.up.pt} & Luís Paulo da Rocha Miranda
\end{tabular}
}
\institute[FEUP/AEDA]{Faculdade de Engenharia da Universidade do Porto \\ Algoritmos e Estruturas de Dados (AEDA) - Turma 5, grupo 3} 
\date[16/nov/2019]{16 de novembro de 2019}
}

\begin{document}
\frame{\titlepage}

\begin{frame}
\frametitle{Problema: Empresa de Transporte de Mercadorias}
\begin{quote}
Modelar um problema recorrendo ao paradigma da orientação por objetos e usar a linguagem C++ para implementar a solução correspondente.
\end{quote}
\begin{quote}
A empresa Transportes SML é especialista no transporte de mercadorias. A empresa possui um número fixo de camiões de diferentes tipos, específicos para o transporte de determinada mercadoria [...]
\end{quote}
\begin{quote}
Os clientes podem requisitar serviços de transporte de mercadorias, o que pode obrigar ao uso de múltiplos camiões por parte da empresa.
\end{quote}
\begin{quote}
Interessa conhecer os valores mensais que a empresa retira dos seus serviços de transporte, para cada um dos tipos de camião e no geral. Deve ser possível monitorar os clientes da empresa e serviços efetuados.
\end{quote}
\end{frame}

\begin{frame}
\frametitle{Solução}
Os utilizadores da aplicação devem fazer login, e possuem permissões diferentes consoante o seu tipo:
\begin{itemize}
	\item Funcionário (\texttt{Employee})
	\begin{itemize}
		\item Gestor (\texttt{Manager})
		\item Condutor (\texttt{Driver})
	\end{itemize}
	\item Cliente (\texttt{Client})
\end{itemize}

Um só tipo de camiões (\texttt{Truck}), que pode carregar uma carga (\texttt{Cargo}) de um dos tipos, cujo preço de transporte varia:
\begin{itemize}
	\item Normal (\texttt{Normal})
	\item Animal (\texttt{Animal})
	\item Refrigerada (\texttt{Refrigerated})
	\item Perigoso (\texttt{Dangerous})
\end{itemize}

Todos os utilizadores podem ver e editar as suas informações pessoais.
O cliente pode ver e editar os seus serviços ativos, ou requisitar um novo. 
O condutor pode ver e editar os seus serviços ativos, bem como ver a lista de camiões disponíveis e as suas estatísticas mensais.
\end{frame}

\begin{frame}
\frametitle{Solução(cont.)}

O gestor pode fazer tudo isto e ainda ver as listas de clientes e funcionários (completas ou filtradas por um argumento específico), com possibilidade de edição para qualquer um destes, além de poder ver os valores mensais de toda a empresa e outras estatísticas.\\
Quando um cliente requisita um serviço, são atribuídos automaticamente condutores e camiões.
Foram utilizados os algoritmos de ordenação mergesort e pesquisa linear para este efeito.
\end{frame}

\begin{frame}
\frametitle{Diagrama de classes}


- diagrama de classes
- diagrama de classes
\end{frame}

\begin{frame}
\frametitle{Estrutura de ficheiros}
\begin{tabular}{p{50mm} p{5mm} p{50mm}}
	Person & & \\	
	{\tiny \lstinputlisting[firstline=6,lastline=7]{./codigo/data/services/services.txt} } & &
\end{tabular}
\end{frame}

\begin{frame}
\frametitle{Estrutura de ficheiros}
clients.txt\\
Guarda a lista de todos os usuários no seguinte formato: Nome; Número de telemóvel; Nome de utilizador; Password; Morada (Rua e número; Código-Postal; Distrito e País); NIF;\\
drivers.txt/managers.txt\\
Guarda a lista de todos os condutores no mesmo formato dos clientes, com uma linha extra equivalente ao salário.
trucks.txt\\
Guarda a lista de todos os camiões da empresa no seguinte formato: Matrícula; Data da matrícula; Combustível (de 0 a 6, de acordo com o tipo); Alcance (em km); Categoria; Número de cargas; Lista de cargas (Peso; Descrição; Preço por km; custo de refrigeração ou nível de perigo, caso se aplique);\\
services.txt\\
Guarda a lista dos serviços no seguinte formato: ID; nome de utilizador (cliente); nome e número de telemóvel do contacto 1; nome e número de telefone do contacto 2; hora de início; hora de fim; distância; tipo de carga; peso da carga; descriçao da carga; número de condutores alocados para esse serviço; Info dos condutores (matricula e nome de utilizador); custo do serviço; preço cobrado ao cliente;
\end{frame}
 
\begin{frame}
\frametitle{Exceções tratadas}
InvalidTimeFormat: chamada quando é introduzido uma hora com o formato errado;\\
InvalidCredentials: chamada quando são introduzidas credenciais erradas (username ou password, durante o login);\\
InvalidTemperatureRange: chamada quando é introduzida uma gama de temperatura não válida;\\
FailedRegex: chamada quando a string introduzida não corresponde a regex;\\
RepeatedId: chamada quando é introduzido um id repetido;\\
%invalidargument: \\
%systemerror: 
\end{frame}

\begin{frame}
\frametitle{Lista de funcionalidades} 
Os clientes podem requisitar serviços, com origem, destino, duração da viagem, ...\\
Mostrar lista de clientes e funcionários; adicionar, editar ou remover clientes e funcionários;\\
Mostrar lista de camiões; adicionar, editar ou remover camiões; \\
Mostrar lista de serviços; adicionar, editar ou remover um serviço;\\
Login para clientes, gestores e condutores;\\
Quando um condutor chega ao destino, tem de ter um contacto que pode, ou não, ser do cliente.\\

\end{frame}

\begin{frame}
\frametitle{Lista de funcionalidades (cont.)} 
Ordenar as diversas listas pelos respetivos argumentos;\\
Filtrar as diversas listas pelos respetivos argumentos;\\
Ver os valores mensais da empresa, tanto totais, como parciais (de um camião, ou de uma categoria de camiões);
\end{frame}

\begin{frame}
\frametitle{Destaque de funcionalidade}
 - falta completar

\end{frame}

\begin{frame}
\frametitle{Dificuldades}
Trabalho de grupo = colaboração + trabalho individual\\
Neste trabalho as classes estavam muito interligadas umas com as outras e com a estrutura da aplicação, pelo que foi dificil compartimentar tarefas e classes/funções
\end{frame}

\begin{frame}
\begin{minipage}[t]{0.33\linewidth}
	\textbf{Diogo Rodrigues}
	{\footnotesize \begin{itemize}
		\item \texttt{utils}: quase tudo, incluindo \texttt{stringregex}, \texttt{mergesort}, \texttt{findif}		
		\item \texttt{Person} e todas as classes derivadas
		\item \texttt{Truck}: reescreveu
		\item \texttt{Cargo}: reescreveu				
		\item I/O ficheiros
		\item \texttt{App}: tabelas, opções das tabelas e mostrar detalhes de entidades
		\item \texttt{Time}, \texttt{Address}, \texttt{Phonenumber}, \texttt{Temperature}, \texttt{VAT}
		\item \texttt{Service} e função de alocação
	\end{itemize} }
\end{minipage}%
\begin{minipage}[t]{0.33\linewidth}
	\textbf{Telmo Baptista}
	{\footnotesize \begin{itemize}
		\item \texttt{utils}: algumas funções
		\item \texttt{App}: input do utilizador; adição, edição e remoção de entidades
		\item \texttt{Cargo}: implementou
		\item \texttt{Person}: alguns métodos
		\item \texttt{Time}: algumas funções
		\item \texttt{Truck}: implementou
	\end{itemize} }
\end{minipage}%
\begin{minipage}[t]{0.33\linewidth}
	\textbf{Luís Miranda}
	{\footnotesize \begin{itemize}
		\item Organização da app
		\item Permissões
		\item Dados
		\item Testes
		\item Apresentação
	\end{itemize} }
\end{minipage}
\end{frame}
\end{document}


